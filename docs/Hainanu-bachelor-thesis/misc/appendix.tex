
\thesisappendix

\chapter{核心脚本文件树}
\section{核心脚本文件树}\label{app:tree}

具体项目参考:\url{https://github.com/hanyijie1/undergraduate_thesis}

\begin{lstlisting}[language=bash, caption={files tree}, label={code:tree_script},
	    extendedchars=true,  % 允许扩展字符集
		inputencoding=utf8x,  % 显式声明UTF-8输入
	]
├── calculator
│   ├── __init__.py
│   ├── n_otc
│   │   ├── cva
│   │   │   ├── __init__.py
│   │   │   ├── __pycache__
│   │   │   ├── spect_calculator.py
│   │   │   ├── trans_amp_calculator.py
│   │   │   └── trans_pro_calculator.py
│   │   ├── __init__.py
│   │   ├── laser_generator.py
│   │   └── __pycache__
│   │       ├── __init__.cpython-311.pyc
│   │       └── laser_generator.cpython-311.pyc
│   ├── n_otc85
│   │   ├── cva
│   │   │   ├── __init__.py
│   │   │   ├── __pycache__
│   │   │   ├── spect_calculator.py
│   │   │   ├── trans_amp_calculator.py
│   │   │   └── trans_pro_calculator.py
│   │   ├── __init__.py
│   │   ├── laser_generator.py
│   │   └── __pycache__
│   │       ├── __init__.cpython-311.pyc
│   │       └── laser_generator.cpython-311.pyc
│   ├── order_test
│   │   ├── cc_generator.py
│   │   ├── err_tester.py
│   │   ├── __init__.py
│   │   ├── math_utils.py
│   │   └── __pycache__
│   │       ├── cc_generator.cpython-311.pyc
│   │       ├── err_tester.cpython-311.pyc
│   │       ├── __init__.cpython-311.pyc
│   │       └── math_utils.cpython-311.pyc
│   ├── otc
│   │   ├── cva
│   │   │   ├── __init__.py
│   │   │   ├── __pycache__
│   │   │   ├── spect_calculator.py
│   │   │   ├── trans_amp_calculator.py
│   │   │   └── trans_pro_calculator.py
│   │   ├── __init__.py
│   │   ├── laser_generator.py
│   │   ├── __pycache__
│   │   │   ├── __init__.cpython-311.pyc
│   │   │   └── laser_generator.cpython-311.pyc
│   │   └── sfa
│   │       ├── __init__.py
│   │       ├── __pycache__
│   │       ├── spect_calculator.py
│   │       ├── trans_amp_calculator.py
│   │       └── trans_pro_calculator.py
│   ├── __pycache__
│   │   └── __init__.cpython-311.pyc
│   └── sc
│       ├── cva
│       │   ├── __init__.py
│       │   ├── __pycache__
│       │   ├── spect_calculator.py
│       │   ├── trans_amp_calculator.py
│       │   └── trans_pro_calculator.py
│       ├── __init__.py
│       ├── laser_generator.py
│       ├── __pycache__
│       │   ├── __init__.cpython-311.pyc
│       │   └── laser_generator.cpython-311.pyc
│       └── sfa
│           ├── __init__.py
│           ├── __pycache__
│           ├── spect_calculator.py
│           ├── trans_amp_calculator.py
│           └── trans_pro_calculator.py
├── const_initializer.py
├── __init__.py
├── math_utils.py
├── __pycache__
│   ├── const_initializer.cpython-311.pyc
│   ├── __init__.cpython-311.pyc
│   └── math_utils.cpython-311.pyc
└── visualizer
├── __init__.py
├── n_otc
│   ├── cva
│   │   ├── amp_contract_visualizer.py
│   │   ├── __init__.py
│   │   ├── __pycache__
│   │   └── trans_pro_spect_visualizer.py
│   ├── __init__.py
│   └── __pycache__
│       └── __init__.cpython-311.pyc
├── n_otc85
│   ├── cva
│   │   ├── __init__.py
│   │   ├── __pycache__
│   │   └── trans_pro_spect_visualizer.py
│   ├── __init__.py
│   ├── laser_visualizer.py
│   └── __pycache__
│       ├── __init__.cpython-311.pyc
│       └── laser_visualizer.cpython-311.pyc
├── order_test
│   ├── err_visualizer.py
│   ├── __init__.py
│   └── __pycache__
│       ├── err_visualizer.cpython-311.pyc
│       └── __init__.cpython-311.pyc
├── otc
│   ├── cva
│   │   ├── __init__.py
│   │   ├── __pycache__
│   │   └── trans_pro_spect_visualizer.py
│   ├── __init__.py
│   ├── laser_visualizer.py
│   ├── __pycache__
│   │   ├── __init__.cpython-311.pyc
│   │   └── laser_visualizer.cpython-311.pyc
│   └── sfa
│       ├── __init__.py
│       ├── __pycache__
│       └── trans_pro_spect_visualizer.py
├── __pycache__
│   └── __init__.cpython-311.pyc
└── sc
├── cva
│   ├── __init__.py
│   ├── __pycache__
│   └── trans_pro_spect_visualizer.py
├── __init__.py
├── laser_visualizer.py
├── __pycache__
│   ├── __init__.cpython-311.pyc
│   └── laser_visualizer.cpython-311.pyc
└── sfa
├── __init__.py
├── __pycache__
└── trans_pro_spect_visualizer.py
\end{lstlisting}


\chapter{积分算法}

\section{亚当斯-巴什福思积分系数生成代码}\label{app:ab_code}
\begin{lstlisting}[language=Python, caption={AB系数生成代码}, label={code:ab_generation}]
import numpy as np
import sympy as sp
from numba import njit, jit

def _lagrange_basis(t, m, points):
    """
	Compute the Lagrange basis polynomial for a given point index.
	
	Constructs the m-th Lagrange basis polynomial evaluated at t, based on a set of interpolation points.
	
	Parameters
	----------
	t : sympy.Expr
	Symbolic variable for evaluation (typically time).
	m : int
	Index of the basis polynomial (0 <= m < len(points)).
	points : list
	list of interpolation points (symbolic or numeric).
	
	Returns
	-------
	sympy.Expr
	The m-th Lagrange basis polynomial evaluated at t.
	"""
	k = len(points)
	numerator = 1
	denominator = 1
	for i in range(k):
		if i != m:
			numerator *= (t - points[i])
			denominator *= (points[m] - points[i])
	return numerator / denominator
def _compute_ab_cc(k):
	"""
	Compute Adams-Bashforth coefficients for numerical integration.
	
	Generates coefficients for the k-step Adams-Bashforth method by integrating
	Lagrange basis polynomials over a uniform grid.
	
	Parameters
	----------
	k : int
	Number of steps for the Adams-Bashforth method (1 <= k <= order).
	
	Returns
	-------
	list
	List of k coefficients for the Adams-Bashforth method.
	"""
	h = sp.symbols('h', positive=True)
	t = sp.symbols('t')
	points = [ -i * h for i in range(k) ]
	coefficients = []
	for j in range(k):
		l_j = _lagrange_basis(t, j, points)
		integral = sp.integrate(l_j, (t, 0, h))
		b_j = integral / h
		coefficients.append(b_j)
	return coefficients
def generate_ab_cc_matrix(k):
	"""
	Generate the Adams-Bashforth coefficient matrix for orders up to k.
	
	Constructs a matrix where each row i contains the coefficients for the (i+1)-step
	Adams-Bashforth integration method.
	
	Parameters
	----------
	k : int
	Maximum order of the Adams-Bashforth method (number of steps).
	
	Returns
	-------
	np.ndarray
	Coefficient matrix of shape (k, k), where row i contains coefficients
	for the (i+1)-step method.
	"""
	coefficients_matrix = np.zeros((k, k))
	for i in range(1, k + 1):
		print("process: k = {}".format(i))
		coefficients_matrix[i - 1, 0:i] = np.array(_compute_ab_cc(i))
	return coefficients_matrix
\end{lstlisting}


\section{积分求解器}\label{app:integral_solver}
\begin{lstlisting}[language=Python, caption={积分求解器}, label={code:integral_solver}]
import numpy as np
from numba import njit

@njit
def integral_solver(
	y_vec: np.ndarray, x_step: np.float64,
	cc_matrix: np.ndarray,
	order_num: int,
	method="ab"
) -> np.ndarray:
    """
	Perform numerical integration using the Adams-Bashforth method on a uniform grid.
	
	Integrates the input array `y_vec` using the multistep Adams-Bashforth method
	with coefficients provided in `cc_matrix`.
	
	Parameters
	----------
	y_vec : np.ndarray
	1D array of integrand values to be integrated.
	x_step : float
	Uniform step size of the grid.
	cc_matrix : np.ndarray
	Matrix of Adams-Bashforth coefficients, where row i contains coefficients
	for the (i+1)-step method.
	order_num : int
	Integration order (number of steps, 2 <= order_num <= 8).
	method : str, optional
	Integration method, currently supports "ab" (Adams-Bashforth). Default is "ab".
	
	Returns
	-------
	np.ndarray
	Array of integrated values, same shape as `y_vec`.
	"""
	if method == "ab":
		y_count = y_vec.size
		integrate_y_vec = np.zeros_like(y_vec)
		for i in range(order_num):
			integrate_y_vec[i + 1] = integrate_y_vec[i]
			for j in range(i + 1):
				integrate_y_vec[i + 1] += x_step * (cc_matrix[i][j] * y_vec[i - j])
	for i in range(order_num, y_count - 1):
		integrate_y_vec[i + 1] = integrate_y_vec[i]
	for j in range(order_num):
		integrate_y_vec[i + 1] += x_step * (cc_matrix[order_num - 1][j] * y_vec[i - j])
	return integrate_y_vec
\end{lstlisting}

\chapter{跃迁概率分布求解}
\section{半经典作用量}\label{app:semi_action}
\begin{lstlisting}[language=Python, caption={半经典作用量}, label={code:semi-action}]
@njit
def derive_semi_action(
	kx: np.ndarray, ky: np.ndarray, kz: float,
	vec_pot_x_vec: np.ndarray, vec_pot_y_vec: np.ndarray,
	ion_pot: float,
) -> np.ndarray:
    """
	Compute time-derivative of semi-classical action S(t).
	
	Parameters
	----------
	kx, ky, kz : ndarray
	Canonical momentum components (a.u.)
	vec_pot_x_vec, vec_pot_y_vec : ndarray
	Vector potential components (time-dependent)
	ion_pot : float
	Ionization potential (Hartree)
	
	Returns
	-------
	ndarray
	dS/dt values computed via:
	dS/dt = 0.5*(p + A(t))² + I_p
	"""
	param_1 = 0.5 * (kx + np.real(vec_pot_x_vec)) ** 2
	param_2 = 0.5 * (ky + np.real(vec_pot_y_vec)) ** 2
	param_3 = 0.5 * kz ** 2 + ion_pot
	semi_action_deriva_vec = param_1 + param_2 + param_3
	return semi_action_deriva_vec
\end{lstlisting}

\section{强场近似跃迁概率分布微分表达式}\label{app:sfa_trans_amp}
\begin{lstlisting}[language=Python, caption={SFA跃迁概率分布微分表达式}, label={code:sfa-trans-amp}]
@njit
def sfa_deriva_trans_amp(
	px: float, py: float, pz: float,
	ele_x_vec, ele_y_vec,
	vec_pot_x_vec: np.ndarray, vec_pot_y_vec:np.ndarray,
	ion_pot: float, atom_ip: float,
	semi_action_vec: np.ndarray,
) -> np.ndarray:
    """
	Compute SFA transition amplitude derivative (dD/dt).
	
	Implements Strong Field Approximation model for ionization rates.
	
	Parameters
	----------
	px, py, pz : float
	Final momentum components (a.u.)
	ele_x_vec, ele_y_vec : ndarray
	Laser electric field components
	vec_pot_x_vec, vec_pot_y_vec : ndarray
	ion_pot, atom_ip : float
	semi_action_vec : ndarray
	Precomputed semi-classical action
	
	Returns
	-------
	ndarray
	Time-dependent transition amplitude derivative
	"""
	qx = px + vec_pot_x_vec
	qy = py + vec_pot_y_vec
	qz = pz + 0.0
	mp_const = - 1j * 2.0 ** (7.0 / 2.0) * (2.0 * ion_pot) ** (5.0 / 4.0) / np.pi
	mp_up = ele_x_vec * qx + ele_y_vec * qy
	mp_down = (qx ** 2 + qy ** 2 + qz ** 2 + 2.0 * np.pi) ** 3.0
	trans_amp_deriva_vec = mp_const * mp_up / mp_down * np.exp(1j * semi_action_vec)
	return trans_amp_deriva_vec
\end{lstlisting}

\section{库伦-沃尔科夫波近似跃迁概率分布微分表达式}\label{app:cva_trans_amp}
\begin{lstlisting}[language=Python, caption={CVA跃迁概率分布微分表达式}, label={code:cva-trans-amp}]
@njit
def cva_deriva_trans_amp(
	px: float, py: float, pz: float,
	ele_x_vec, ele_y_vec,
	vec_pot_x_vec: np.ndarray, vec_pot_y_vec:np.ndarray,
	ion_pot: float, atom_ip: float,
	semi_action_vec: np.ndarray,
) -> np.ndarray:
    """
	Compute CVA transition amplitude derivative (dD/dt).
	
	Implements Coulomb-Volkov Approximation including Coulomb corrections.
	
	Parameters match sfa_deriva_trans_amp with additional:
	atom_ip : float
	Atomic ionization potential for Coulomb correction
	"""
	qx = px + vec_pot_x_vec
	qy = py + vec_pot_y_vec
	qz = pz + 0.0
	pr = np.sqrt(px ** 2 + py ** 2 + pz ** 2)
	qr = np.sqrt(qx ** 2 + qy ** 2 + qz ** 2)
	vv = 1.0 / pr
	a1 = vv
	dd = atom_ip ** 2 + (qx ** 2 + qy ** 2 + qz ** 2)
	ss = - (px * qx + py * qy + pz * qz) - 1j * atom_ip * pr
	uu = 2.0 * ss / dd
	aa = 1.0 + uu
	b1 = 2.0 * (1j * pr + atom_ip * uu)
	l01 = 2.0 * atom_ip - 1j * a1 * b1 / aa
	int_r = 4.0 * np.pi * aa ** (-1j * a1) * l01 / dd ** 2
	nt_p = np.exp(np.pi * vv / 2.0) * _lanczos_gamma(1.0 - 1j * vv)
	ds_dt = (0.5 * qr ** 2 + ion_pot)
	mp_const = - 1j * (2.0 * np.pi) ** (-1.5) / np.sqrt(4.0 * np.pi)
	trans_amp_deriva_vec = mp_const * nt_p * int_r * np.exp(1j * semi_action_vec) * ds_dt
	return trans_amp_deriva_vec
\end{lstlisting}

\section{非正交激光跃迁振幅求解示范}\label{app:notc_cva_trans_amp}
\begin{lstlisting}[language=Python, caption={非正交激光跃迁振幅求解示范}, label={code:notc cva trans amp}]
from src.const_initializer import ConstInitializer
from src.math_utils import *
import numpy as np
from numba import jit, prange
import multiprocessing
import h5py
import pandas as pd

@jit(nopython=True)
def _jit_semi_action(
	kx: np.ndarray, ky: np.ndarray, kz: float,
	vec_pot_x_vec: np.ndarray, vec_pot_y_vec: np.ndarray,
	ion_pot: float,
	t_step: np.float64,
	cc_matrix: np.ndarray,
	trans_amp_order: int,
) -> np.ndarray:
	"""
	Compute the semi-classical action by integrating its time derivative.
	
	Optimized with Numba for performance, this function calculates the semi-classical
	action S(t) by integrating dS/dt using the Adams-Bashforth method.
	
	Parameters
	----------
	kx, ky : np.ndarray
	Canonical momentum components in x and y directions (atomic units).
	kz : float
	Canonical momentum component in z direction (atomic units, typically 0).
	vec_pot_x_vec, vec_pot_y_vec : np.ndarray
	Time-dependent vector potential components in x and y directions (atomic units).
	ion_pot : float
	Ionization potential (Hartree, atomic units).
	t_step : float
	Time step size for integration (atomic units).
	cc_matrix : np.ndarray
	Adams-Bashforth coefficient matrix, where row i contains coefficients for
	the (i+1)-step method.
	trans_amp_order : int
	Integration order for the Adams-Bashforth method (2 <= order <= 8).
	
	Returns
	-------
	np.ndarray
	Array of semi-classical action values S(t) over the time grid.
	
	Notes
	-----
	The semi-classical action is computed as:
	.. math::
	S(t) = \int \frac{dS}{dt} \, dt, \quad \frac{dS}{dt} = \frac{1}{2} (p + A(t))^2 + I_p
	where :math:`p` is the canonical momentum, :math:`A(t)` is the vector potential,
	and :math:`I_p` is the ionization potential.
	"""
	semi_action_deriva_vec = derive_semi_action(
	kx, ky, kz,
	vec_pot_x_vec, vec_pot_y_vec,
	ion_pot,
	)
	semi_action_vec = integral_solver(semi_action_deriva_vec, t_step, cc_matrix, trans_amp_order)
	return semi_action_vec

@jit(nopython=True, parallel=True,)
def _jit_trans_amp(
	thread_count,
	p_count: int, theta_count: int, alpha_count: int,
	p_vec: np.ndarray, theta_vec: np.ndarray,
	ion_pot: float, atom_ip: float,
	ele_x_matrix: np.ndarray, ele_y_matrix: np.ndarray,
	vec_pot_x_matrix: np.ndarray, vec_pot_y_matrix: np.ndarray, t_step: np.float64,
	cc_matrix: np.ndarray,
	trans_amp_order: int,
) -> np.ndarray:
	"""
	Compute the Coulomb-Volkov Approximation (CVA) transition amplitude.
	
	Optimized with Numba for parallel execution, this function calculates the
	transition amplitude for a range of momentum, angle, and polarization parameters.
	
	Parameters
	----------
	thread_count : int
	Number of parallel threads (typically set to CPU core count).
	p_count : int
	Number of momentum grid points.
	theta_count : int
	Number of angular grid points (theta, azimuthal angle).
	alpha_count : int
	Number of polarization angles (for N-OTC configuration).
	p_vec : np.ndarray
	1D array of momentum magnitudes (atomic units).
	theta_vec : np.ndarray
	1D array of azimuthal angles (radians, 0 to 2π).
	ion_pot : float
	Ionization potential (Hartree, atomic units).
	atom_ip : float
	Atomic ionization potential for Coulomb correction (atomic units).
	ele_x_matrix, ele_y_matrix : np.ndarray
	2D arrays of electric field components (shape: [t_count, alpha_count], atomic units).
	vec_pot_x_matrix, vec_pot_y_matrix : np.ndarray
	2D arrays of vector potential components (shape: [t_count, alpha_count], atomic units).
	t_step : float
	Time step size for integration (atomic units).
	cc_matrix : np.ndarray
	Adams-Bashforth coefficient matrix, where row i contains coefficients for
	the (i+1)-step method.
	trans_amp_order : int
	Integration order for the Adams-Bashforth method (2 <= order <= 8).
	
	Returns
	-------
	np.ndarray
	3D array of transition amplitudes (shape: [p_count, theta_count, alpha_count],
	dtype: complex128).
	
	Notes
	-----
	The transition amplitude is computed by integrating the CVA derivative:
	.. math::
	D(p, \theta, \alpha) = \int \frac{dD}{dt} \, dt
	where :math:`\frac{dD}{dt}` is provided by `cva_deriva_trans_amp`, and the integration
	uses the Adams-Bashforth method.
	"""
	# target
	trans_amp_display_arr = np.zeros(
		(p_count, theta_count, alpha_count),
		dtype=np.complex128,
	)
	single_progress_count = 0
	for i in prange(p_count):
	single_progress_count += 1
	print("Current task progress", single_progress_count / (p_count / thread_count))
	for j in range(theta_count):
	px = p_vec[i] * np.cos(theta_vec[j])
	py = p_vec[i] * np.sin(theta_vec[j])
	pz = 0.0
	for k in range(alpha_count):
		ele_x_vec, ele_y_vec = ele_x_matrix[:, k], ele_y_matrix[:, k]
		vec_pot_x_vec, vec_pot_y_vec = vec_pot_x_matrix[:, k], vec_pot_y_matrix[:, k]
		semi_action_vec = _jit_semi_action(
			px, py, pz,
			vec_pot_x_vec, vec_pot_y_vec,
			ion_pot,
			t_step,
			cc_matrix,
			trans_amp_order,
		)
		trans_amp_deriva_vec = cva_deriva_trans_amp(
			px, py, pz,
			ele_x_vec, ele_y_vec,
			vec_pot_x_vec, vec_pot_y_vec,
			ion_pot, atom_ip,
			semi_action_vec,
		)
		trans_amp_part_vec = integral_solver(
			trans_amp_deriva_vec,
			t_step,
			cc_matrix,
			trans_amp_order,
		)
		# save
		trans_amp_display_arr[i, j, k] = trans_amp_part_vec[- 1]
		return trans_amp_display_arr
	
class TransAmpCalculator(ConstInitializer):
	"""
	Calculate and store transition amplitudes for strong-field physics simulations.
	
	Inherits from `ConstInitializer` to provide configuration parameters and
	implements methods for computing and saving Coulomb-Volkov Approximation (CVA)
	transition amplitudes for the N-OTC configuration.
	
	Attributes
	----------
	thread_count : int
	Number of CPU threads for parallel computation.
	semi_action_vec : np.ndarray
	Array to store semi-classical action values (shape: [t_count]).
	ele_x_matrix, ele_y_matrix : np.ndarray
	Electric field components (shape: [t_count, alpha_count], atomic units).
	vec_pot_x_matrix, vec_pot_y_matrix : np.ndarray
	Vector potential components (shape: [t_count, alpha_count], atomic units).
	cc_matrix : np.ndarray
	Adams-Bashforth coefficient matrix for numerical integration.
	
	Methods
	-------
	_calculate_visual_trans_amp()
	Compute transition amplitudes for visualization.
	save_visual_trans_amp()
	Save computed transition amplitudes to an HDF5 file.
	"""
	def __init__(self,):
		"""
		Initialize the transition amplitude calculator.
		
		Loads laser field data and Adams-Bashforth coefficients from files and
		sets up parameters inherited from `ConstInitializer`.
		"""
		super().__init__()
		self.thread_count = multiprocessing.cpu_count()
		self.semi_action_vec = np.zeros(self.t_count)
		my_df = pd.read_hdf(self.n_otc_laser_data_file, key=self.laser_name)
		self.ele_x_matrix, self.ele_y_matrix \
		= my_df[self.ele_x_table_label].values.astype(np.float64).reshape(self.t_count, self.alpha_count), \
		my_df[self.ele_y_table_label].values.astype(np.float64).reshape(self.t_count, self.alpha_count)
		self.vec_pot_x_matrix, self.vec_pot_y_matrix \
		= my_df[self.vec_pot_x_table_label].values.astype(np.float64).reshape(self.t_count, self.alpha_count), \
		my_df[self.vec_pot_y_table_label].values.astype(np.float64).reshape(self.t_count, self.alpha_count)  # 显式转化为arr, 以便numba优化。
		with h5py.File(self.order_test_data_file_path, mode="r") as f:
		self.cc_matrix = f[self.ab_cc_name][:].astype(np.float64)
	
	def _calculate_visual_trans_amp(self) -> np.ndarray:
		"""
		Compute transition amplitudes for the N-OTC configuration.
		
		Calls the Numba-optimized `_jit_trans_amp` function to calculate the
		Coulomb-Volkov Approximation transition amplitudes.
		
		Returns
		-------
		np.ndarray
		3D array of transition amplitudes (shape: [p_count, theta_count, alpha_count],
		dtype: complex128).
		"""
		trans_amp_display_arr = _jit_trans_amp(
		self.thread_count,
		self.p_count, self.theta_count, self.alpha_count,
		self.p_vec, self.theta_vec,
		self.ion_pot, self.atom_ip,
		self.ele_x_matrix, self.ele_y_matrix,
		self.vec_pot_x_matrix, self.vec_pot_y_matrix, self.t_step,
		self.cc_matrix,
		self.trans_amp_order,
		)
	return trans_amp_display_arr
	
	def save_visual_trans_amp(self, ) -> None:
		"""
		Save computed transition amplitudes to an HDF5 file.
		
		Stores the transition amplitude data in the N-OTC CVA data file, overwriting
		any existing dataset with the same name.
		
		Notes
		-----
		The data is saved in the HDF5 file at `self.n_otc_cva_data_file` under the
		dataset name `self.trans_amp_name`.
		"""
		with h5py.File(self.n_otc_cva_data_file , "a") as collect_file:
		if self.trans_amp_name in collect_file:  # 如果存在,删除该数据集
			del collect_file[self.trans_amp_name]
		trans_amp_display_dset = collect_file.create_dataset(
			self.trans_amp_name,
			(self.p_count, self.theta_count, self.alpha_count),
			dtype=np.complex128,
		)
		trans_amp_display_dset[:] = self._calculate_visual_trans_amp()
\end{lstlisting}