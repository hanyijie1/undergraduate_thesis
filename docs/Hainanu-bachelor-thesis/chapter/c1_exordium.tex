
\chapter{引\hspace{6pt}言}

\section{研究工作的背景与意义}
飞秒激光因其超短脉冲特性(时间尺度为 $10^{-15}$ 秒)和高聚焦能力是研究原子与分子强场相互作用的理想工具。20世纪80年代,非线性光学领域的进展为飞秒激光的应用奠定了基础,研究者们开始探索多光子电离和高次谐波生成(High-Harmonic Generation, HHG)等现象\citing{McPherson1987HHG}。1993年,Paul B. Corkum提出的高次谐波生成三步模型(隧穿电离、电子加速、电子复合)\citing{Corkum1993}解释了HHG的微观机制,成为强场物理的理论基石。

飞秒激光技术的飞跃在1990年代末至2000年代初。少循环脉冲技术的突破使得科学家能够精确控制载波包络相位(Carrier-Envelope Phase, CEP)\citing{Xu1996}\citing{Baltuska1999},此进展为研究电子波包的时域干涉提供了关键控制维度。阿秒脉冲产生技术的突破\citing{Trieloff2003},使研究者得以在亚周期时间尺度上解析电子动力学。通过设计双色场的时间延迟,研究者成功观测到电离电子周期间(inter-cycle)与周期内(intra-cycle)干涉条纹\citing{Arbo2012Doubly},这些干涉特征直接反映了电子在连续态中的相位积累过程。随着时空分辨技术的革新,研究视角逐步突破正交场体系的局限。通过构建非正交双色场(Non-Orthogonal Two-Color, NOTC),研究发现其时空干涉具有独特的各向异性特征:激光偏振夹角的变化可导致动量空间干涉条纹的对称性破缺\citing{Richter2016Ionization},而库仑势则通过相位调制影响再碰撞电子的路径选择\citing{Jin2022Control}。这些现象揭示了时域干涉与空间构型的耦合机制,为光场调控电子动力学开辟了新的研究维度。近年来,通过优化双色场时空参数,研究团队实现了阿秒条纹测量中的干涉增强效果\citing{Xie2015TwoDimensional},这一进展对阿秒计量学的发展具有重要理论价值和应用潜力。

综上所述,时空干涉机制的研究处于强场物理的核心前沿,随着库仑势在时域干涉中的相位调控作用被更精确地解析,阿秒尺度实现电子运动的控制愈发主动。

\section{国内外研究历史与现状}
然而,近年来,利用相等峰值强度的OTC场探究时空干涉机制成为研究热点。例如,通过调整双色激光的相对相位可以增强或抑制多重干涉结构,揭示了前向散射电子在强场物理中的库仑效应的贡献\citing{Jin2022Control};通过使用周期形状的正交偏振双色激光场研究电子波包干涉和二维干涉测量方法、分析光电子动量分布中的干涉图案,实现了阿秒级时间分辨率和埃级空间分辨率\citing{Xie2015TwoDimensional};通过使用三维含时薛定谔方程的精确解和半经典方法(如量子轨迹蒙特卡洛模拟和库仑校正强场近似)研究探讨了正交偏振双色激光场中非绝热亚周期电子动力学发现了非绝热效应和库仑势在超快电子过程中起关键作用\citing{Geng2015Nonadiabatic}。

相比之下,非正交双色激光场的研究因其理论复杂性和实验实现难度较高相对较少,但是非正交双色激光场在高次谐波生成与阿秒科学方面也具有深远意义。例如,使用低频线性偏振场和高频椭圆偏振场的双色泵浦方法可以实现阿秒脉冲的生成\citing{Taranukhin2004Attosecond};两色平行线性偏振场的组合可以缩短相位匹配窗口,从而可以生成更短的阿秒脉冲,这暗示非正交配置可能进一步优化这一过程\citing{Chen2017Influence}。

等强度OTC脉冲和NOTC产生的电子动量分布可用强场近似(Strong-Field Approximation, SFA)理论解释,该理论忽略了离子库仑势对光致电离电子的影响\citing{Arbo2008Coulomb}。此近似基本合理,但是在某些OTC相对相位下,SFA模拟与实验观测存在显著差异\citing{Richter2016Ionization}。由此一种改进的SFA理论——库仑-沃尔科夫波近似(Coulomb-Volkov Approximation,CVA)\citing{Arbo2008Coulomb}被用于探索原子电离中的库仑效应。该理论采用包含库仑修正的库仑-沃尔科夫态(而非SFA中的沃尔科夫态)描述库伦势和电场的耦合相互作用。

\section{本文的主要贡献与创新}
本文通过SFA、CVA理论研究了单色激光场(Single-Color, SC)、等强度正交双色激光场(OTC)和等强度线性偏振非正交双色激光场(NOTC)产生的电子动量分布和能谱结构。

在激光源为SC和OTC的时候,本文重点阐明了库仑势在周期内和周期间对电子动量分布中的时域干涉作用和能谱结构。结果表明,与SFA模拟相比,CVA模拟的电子动量分布对相位库仑修正一方面增强了前向再散射电子的贡献,另一方面削弱了库仑势存在时直接电子的贡献。例如,OTC脉冲相对相位 $\phi=\frac{\pi}{2}$ 时的结果,尽管和\citing{Yu2016Coulomb}中在激光场包络的选取上有一定的不同,其多周期的干涉图案和库伦聚焦的作用却有高度的一致性;同时周期内的能谱对多周期能谱的调制作用也可以和\citing{Arbo2011Doubly}中的结果相符。

在激光源为NOTC的时候,本文重点单独叙述CVA近似下激光夹角 $\alpha$ 为 $85^\circ$ 情况OTC的对比情况,并对 $\alpha$ 角进行参数扫描,对比其能谱和不同动量下各个角度的跃迁概率密度。

\section{本论文的结构安排}
本文的章节结构安排如下:第二节简要介绍Adams-Bashforth、SFA和CVA理论,以及其能谱计算方式;随后第三节首先进行Adams-Bashforth对数值积分阶数的选取和误差分析,再展示SC、OTC、NOTC脉冲对相位差 $\phi = \frac{\pi}{2}$ 下电子动量分布的SFA与CVA模拟,并对NOTC激光夹角参数 $\alpha$ 进行扫描绘谱;最后第四节给出结论。除非特殊说明,全文采用原子单位制(Atomic Units, a.u.)。
