\chapter{全文总结与展望}

\section{全文总结}

本文在融合变阶亚当斯-巴什福思积分算法的并行计算框架下,基于强场近似与库仑-沃尔科夫波近似理论,针对单色线偏振场、正交双色场及非正交双色场体系,完成了包含时域干涉相关的跃迁振幅计算、动量谱结构分析和能谱分析。结果表示,库仑势聚焦低动量电子并复杂化高动量干涉图案,吸引低能电子并形成高能网格结构,通过时域干涉显著调控电子角分布;非正交双色场中,激光夹角变化影响再碰撞动力学,诱导低动量双峰与能谱共振。本研究验证基于时域干涉调控的库仑-沃尔科夫波近似模型在时域干涉主导的周期间与周期内电离过程中可以显著提升特别是高能区光电子动量谱的预测精度,并揭示了时域干涉诱导的各向异性,为强场物理建立了系统的分析方案。

\section{后续工作展望}
上述研究内容可以在多电子系统、电子相关性和分子轨道效应多方面进行扩展。结合非绝热理论优化CVA模型,可联合实验描述超短脉冲动力学,以探索非正交双色场在阿秒脉冲与高次谐波中的应用\citing{Geng2015}。对于更复杂模拟需求,可考虑cython进一步编译策略和缓存机制优化或C++或julia等高性能语言重构,以及添加GPU并行。